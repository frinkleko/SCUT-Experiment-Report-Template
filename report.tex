\documentclass[a4paper,12pt]{ctexart}
    \title{\fontsize{30pt}{\baselineskip}\selectfont\textbf{?}}
    \author{}
    \date{}
    \usepackage{amsmath}
    \usepackage{fancyhdr}    % 页眉页脚格式控制
    \usepackage{caption}
    \usepackage{titlesec}
    \usepackage{indentfirst}
    \usepackage{minted}
    \usepackage{listings}    % 代码插入
    \usepackage{xcolor}        % 颜色
    \usepackage{fontspec}    % 选字体
	\usepackage[T1]{fontenc}
	\usepackage{booktabs} % 三线表
    \usepackage{bookmark}
    \usepackage{subfigure}
    \usepackage{clrscode}
    \usepackage{hyperref}    % 交叉引用
                            % \aref{LABEL_HERE}
    \usepackage{makecell}    % 表格特殊单元格设计
    \usepackage{multirow}
    \usepackage{multicol}
    \usepackage{longtable}
    \usepackage{dirtree}
    \usepackage{amssymb}
    \usepackage{mdwlist}
    \usepackage{graphicx}
    \usepackage{geometry}
	\usepackage[inkscapelatex=false]{svg} % 支持svg图片,当然推荐您直接插入pdf图片

    \hypersetup{pdftitle={PDFTITLE},    %pdf文件属性
                pdfauthor={PDFAUTHOR}}
    \geometry{left=3.17cm,right=3.17cm,top=2.54cm,bottom=2.54cm}
    \setcounter{tocdepth}{3}
    
    \newfontfamily\consolas{Consolas}    % 引入 Monaco 字体    
	\pagestyle{fancy}	% 页眉 及 页脚
		\lhead{}
		\chead{}
		\rhead{\textsl{South China University of Technology}}
		\lfoot{}
		\rfoot{}
		\cfoot{\thepage}
	\renewcommand{\headrulewidth}{0pt}
    \newcommand{\tabincell}[2]{\begin{tabular}{@{}#1@{}}#2\end{tabular}}    % 单元格内建表
    
    \setlength{\parindent}{2em}
 
    \captionsetup[figure]{labelfont={bf},labelformat={simple},labelsep=period,name={图}}
    \captionsetup[table]{labelfont={bf},labelformat={simple},labelsep=period,name={表}}
    
    \newcommand{\sectionfontsize}{\fontsize{15pt}{18pt}\selectfont}
    \newcommand{\parfontsize}{\fontsize{12pt}{18pt}\selectfont}

    \ctexset{
        % 修改 section。
        section={   
            name={实验,:},
            number={\chinese{section}},
            format=\heiti\bfseries\centering\zihao{-2} % 设置 section 标题为黑体、右对齐、小4号字
        },
        % 修改 subsection。
        subsection={   
            name={,、},
            number={\chinese{subsection}},
            format=\heiti\bfseries\zihao{4} % 设置 subsection 标题为黑体、5号字
        },
        % 修改 subsection。
        subsubsection={   
            name={,、},
            number={\arabic{subsubsection}},
            format=\heiti\bfseries\zihao{-4} % 设置 subsection 标题为黑体、5号字
        }
    }
    \fancyhead{}
    
    \hypersetup{hidelinks}
    \def\figureautorefname{图}%
    \def\tableautorefname{表}%
    \newcommand{\aref}[1]{\textbf{\autoref{#1}}}
    
    \newcounter{RomanNumber}
    \newcommand{\romanNumber}[1]{\setcounter{RomanNumber}{#1}\Roman{RomanNumber}}    
    
    \hfuzz=\maxdimen    % 忽略宽度限定
    \tolerance=10000
    \hbadness=10000
    
    \setlength{\baselineskip}{30pt}
    \renewcommand{\contentsname}{\hspace*{\fill}目\quad 录\hspace*{\fill}}
    
\begin{document}
%%--------------------------FORE_PAGE---------------------------%%

%制作封面
\begin{titlepage}
    \begin{center}
        \par
        \centerline{\includegraphics[scale=1.2]{images/logo.png}} %插入图片
        \par
        \vskip 5cm
        \lishu \fontsize{50}{20} 实\quad 验\quad 报\quad 告
        \vskip 10cm

        \begin{tabular}{l}
            \songti \zihao{-2} \bfseries 课程名称:
            \quad \\
            \songti \zihao{-2} \bfseries 学生姓名:
            \quad \\
            \songti \zihao{-2} \bfseries 学生学号:
            \quad \\
            \songti \zihao{-2} \bfseries 学生专业:
            \quad \\
            \songti \zihao{-2} \bfseries 开课学期 :
            \quad \\
            \songti \zihao{-2} \bfseries  提交日期:\today
        \end{tabular}
    \end{center}
\end{titlepage}

% 生成目录
\newpage
% \pagestyle{empty}
\begin{center}
    \tableofcontents
\end{center}

%%--------------------------MAIN_BODY--------------------------%%
\newpage

\setcounter{page}{1}

\section{XXXXXX}
\subsection{实验目的}
这里写实验目的。

\subsection{实验题目}
这里写实验原理。可以使用公式、图表等辅助说明。

\subsection{实验内容}
这里写实验步骤与过程。
\subsubsection{代码插入演示}

\begin{minted}{python}
    import numpy as np
    import matplotlib.pyplot as plt
    from sklearn import datasets, linear_model
    from sklearn.metrics import mean_squared_error, r2_score

    # Load the diabetes dataset
    diabetes = datasets.load_diabetes()

    # Use only one feature
    diabetes_X = diabetes.data[:, np.newaxis, 2]

    # Split the data into training/testing sets
    diabetes_X_train = diabetes_X[:-20]
    diabetes_X_test = diabetes_X[-20:]
\end{minted}

当然你也可以尝试用listing展示minted,效果如下
\begin{listing}[!ht]
    \begin{minted}{c}
    #include <stdio.h>
    int main() {
       printf("Hello, World!"); /*printf() outputs the quoted string*/
       return 0;
    }
    \end{minted}
    \caption{Hello World in C}
    \label{listing:2}
\end{listing}

\begin{listing}[!ht]
    \begin{minted}{lua}
    function fact (n)--defines a factorial function
      if n == 0 then
        return 1
      else
        return n * fact(n-1)
      end
    end
    
    print("enter a number:")
    a = io.read("*number") -- read a number
    print(fact(a))
    \end{minted}
    \caption{Example from the Lua manual}
    \label{listing:3}
\end{listing}
\noindent\texttt{minted} makes a nice job of typesetting listings \ref{listing:2} and \ref{listing:3}.
\subsubsection{三线表示例}
\begin{table}[htbp]
    \centering
    \caption{three-line table}
    \begin{tabular}{cccc}
        \toprule  % 顶部线
        1   & 2   & 3   & 4   \\
        \midrule  % 中部线
        0.1 & 0.2 & 0.3 & 0.4 \\
        \bottomrule  % 底部线
    \end{tabular}
\end{table}


\subsection{实验结果}
这里展示实验结果。可以使用图表等形式展示。

\subsection{实验结论}
这里对实验结果进行分析与讨论,给出实验结论。

\newpage
\section{XXXXXX}

\end{document}
